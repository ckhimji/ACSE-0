\documentclass[11pt]{article}
%\usepackage[a3paper]{geometry}
%\usepackage[]{babel}
\usepackage[utf8]{inputenc}
\usepackage[T1]{fontenc}
\usepackage{fancyhdr}
\usepackage{graphicx}
\usepackage{amsmath}
\usepackage{amsfonts}
\usepackage{enumerate}
\PassOptionsToPackage{hyphens}{url}\usepackage{hyperref}


\topmargin=0cm \oddsidemargin=0cm \evensidemargin=0cm \textheight=21cm
\textwidth=16cm \headheight=15pt \footskip=35pt

%\topmargin=0cm \oddsidemargin=0cm \evensidemargin=0cm \textheight=35cm
%\textwidth=25cm \headheight=15pt \footskip=35pt

\pagestyle{fancyplain}
\fancyhead{} % clear all header fields
\lhead{ACSE}
\rhead{Required software}
\fancyfoot{} % clear all footer fields
\cfoot{\thepage}

\begin{document}

%headers
\begin{center}
{\bf \Large Required software and brief installation guide for the ACSE MSc course } \\ \today
\end{center}
\hrule
\vspace*{1cm}

% beginning of the content

\section{Foreword}

Laptops are provided in their original box, which means you are responsible for installing them and using them according to College information security regulations (\url{https://www.imperial.ac.uk/admin-services/secretariat/college-governance/charters/policies-regulations-and-codes-of-practice/information-security-/policy/it-resources/}). 
In particular, you are responsible for installing and configuring the appropriate security software (\url{https://www.imperial.ac.uk/admin-services/ict/self-service/be-secure/}).


\section{List of required software}

Here is a list of  software that you need to have installed on your laptop for the first four modules of the course.
\footnote{In some cases, alternatives to our recommended solutions exist, but usually require more knowledge to maintain and might require some tweaking for some modules examples. 
Therefore, if you choose to not follow our recommendations, it is at your own risk, and no formal support will be provided.}

\begin{itemize}
  \item Windows 10  64bit Pro, Enterprise or Education Editions (1607 Anniversary Update, Build 14393 or later) OR OSX  El Capitan 10.11 or later OR Ubuntu 16.04 LTS or later (other unix based systems will probably be fine as well provided you know how to use them)
  \item Anaconda (python 3.6 version)
  \item Firefox or Chrome
  \item Docker
  \item git
  \item Putty and WinSCP
\end{itemize}


\section{GitHub account}

The GitHub platform will be widely used throughout the course, both to distribute material and to host your assignments.

If you do not already have one, create a GitHub account (\url{https://github.com/join}).
In order to create private repositories (where you can store your assignments without making them visible to others), apply to the student github status:\\ 
\url{https://help.github.com/articles/applying-for-a-student-developer-pack/}.

Send us your GitHub username so we can give you access to course material.


\section{Installation guide}

\subsection{Start-up}

You are free to install the laptop however you want. We strongly advise declining to use your Microsoft account (either College one or personal one) for now. You can use the username you want, provided it fits in College regulations. 

Please do not install standard operating system updates during the IT clinic hours, as they can require several long reboot cycles.

\subsection{WiFi}
The WiFi network you should connect to is "Imperial-WPA" with your Imperial username (username@ic.ac.uk or ic\\username) and password.

\subsection{Windows 10 Education}

Licenses are provided for free for Imperial College students. Check your licence in Settings -> System -> About. If it is Windows Home Edition, upgrade to an Education Edition:
\begin{itemize}
  \item Go to \url{http://www.onthehub.com/microsoft-windows-10-education-for-students/?utm_source=ms-student-page&utm_medium=microsoft-site&utm_campaign=windows10}
  \item Get a licence key (select \"Imperial College London - Microsoft Imagine Premium\" as Department)
  \item Do not download the disk image (.iso file)
  \item Go in windows settings -> System -> About -> Upgrade your edition of Windows
  \item Type in the licence key
  \item Follow instructions and reboot
\end{itemize}

\subsection{Anaconda}

Anaconda distribution is a python distribution with a lot of pre-packaged libraries. 
Download and install should be straightforward from this page: \url{https://www.anaconda.com/download}. 
Be careful to select the 64-Bit Python 3.6 version.

\subsection{Docker}

Docker is a containerization software that we will use to distribute some codes that do not natively run on Windows. 
It can be downloaded from this page \url{https://docs.docker.com/docker-for-windows/install/}. 
Follow the sign up and installation instructions above. If virtualization has been disabled in your computer's BIOS, you may need to reenable it (\url{https://bce.berkeley.edu/enabling-virtualization-in-your-pc-bios.html}). This may require rebooting your computer.

\subsection{git}

Git is a version control software that will be widely used throughout the course. 
The windows version can be downloaded and installed from \url{https://git-scm.com/download/win}. 
If asked for a default text editor and you know none of the options, choose nano.

\subsection{Putty and winSCP}

These two utilities allow you to connect to remote computers. 
Installation is extremely straightforward from 
\url{https://www.chiark.greenend.org.uk/~sgtatham/putty/latest.html} 
and \url{https://winscp.net/eng/download.php}.

\subsection{Web browsers}

A significant amount of course material will be delivered through Jupyter Notebooks, which are run in a web browser. 
Jupyter Notebooks may not work properly with Microsoft IE or Edge. 
Therefore it is recommended to use either Chrome or Mozilla Firefox on Windows.  
\url{https://www.google.com/chrome/}
\url{https://www.mozilla.org/en-GB/firefox/new/} 

\subsection{JupyterHub}

A cloud-based solution for running Jupyter notebooks will be used in some lectures. Please log-in once at the following address and confirm it worked by the end of the week.
\url{https://ese-jhub.westeurope.cloudapp.azure.com/}


\subsection{Windows Subsystem for Linux}

The Windows Subsystem for Linux (\url{https://docs.microsoft.com/en-us/windows/wsl/about}) might be an alternative to using Windows versions of the previously mentioned software. 
However there is no guarantee it will work for all the lectures and projects, and it is therefore not recommended except for recreational exploration.


\section{Other useful software}

The following software will be useful later in the course:
\begin{itemize}
  \item Microsoft Office for admin purposes (licence provided by the College, \url{http://www.imperial.ac.uk/admin-services/ict/training-and-resources/microsoft-office-365/})
  \item LaTeX, to write scientific reports (\url{https://www.latex-project.org/get/})
  \item Paraview (\url{https://www.paraview.org/download/})
  \item Slack, for team work (\url{https://slack.com})
\end{itemize}


\subsection{Antivirus}
Windows 10 comes with built in antivirus software called Windows Defender. It is turned on by default, updates automatically and scans files as you use them. 
Imperial College uses Symantec Endpoint Protection. If you would prefer to use that, then:
\begin{itemize}
  \item Go to \url{https://www.imperial.ac.uk/ict/services/software/shop/index.asp} and login with your College username and your College password.
  \item Click on Select and Download Software > Anti-Virus.
  \item Tick the box next to AntiVirus for Windows > Accept > Accept > Complete Order.
  \item Follow the instructions in the e-mail you receive to install Symantec Endpoint Protection for Windows. 
\end{itemize}

\subsection{Microsoft Office}
\begin{itemize}
  \item Go to \url{https://www.imperial.ac.uk/office365} and login with your College username@ic.ac.uk and your College password.
  \item Click on the \"waffle\" icon in the top left corner of the browser window and then on Office 365.
  \item Click on the Install Office apps button on the top right of the browser window, and then Office 2016.
  \item Click on Run (or Save) to being downloading the installer. Once complete, click Yes to start installing.
  \item Once the install is complete, login to Office 365 with your username@ic.ac.uk credentials.
\end{itemize}

\subsection{OneDrive sync client}
The OneDrive Sync Client, which comes with Windows 10, lets you connect and synchronize files from your College OneDrive for Business to your laptop. We recommend that you store your documents on OneDrive for Business as you can access them easily from any device and they are backed up in the cloud.
More info available about One Drive available at \url{http://www.imperial.ac.uk/admin-services/ict/self-service/connect-communicate/office-365/features/onedrive-for-business/}

\subsection{AppsAnywhere} 
\begin{itemize}
  \item The Software Hub provides you with access to lots of free software.
  \item Go to \url{https://softwarehub.imperial.ac.uk} and login with your college credentials.
  \item Click on It’s my first time using AppsAnywhere on this device. Let’s go – to download the AppsAnywhere client.
  \item Run the downloaded program, and then accept all the defaults.
  \item AppsAnywhere will automatically launch and install the Cloudpaging player, if required.
  \item Go back to your web browser and on the Software Hub page click on Done.
  \item If prompted, click Allow to let your browser interact with the AppsAnywhere client.
  \item The AppsAnywhere client will open and, once validated, you are ready to launch applications from the Software Hub webpage.
  \item Hover over an application and click on Launch, Download or Visit Website.
\end{itemize}

\paragraph*{Things to Note:}
\begin{itemize}
  \item Large applications can take a while to launch, especially when you run them for the first time. A slow internet connection can also impact the time it takes to launch. Be patient. The player will show you the progress.
  \item The software runs just like it would if it was installed on the machine. So it uses the local hardware such as the processor, memory, and hard disk. Therefore if it’s running slowly, it could be due to the hardware spec of the machine you’re running it on.
  \item You can run the software on any campus computer. So if, for example, you’re at South Kensington, you can logon to the machines in the Central Library, click on the Software Hub link on the desktop, and you’ll be able to run your software from there.
  \item You can run the software though Application Jukebox on any personal machine. This is useful if you have more than one machine. 
\end{itemize}


\section{Getting Help}

If you need any help, please contact the ICT Service Desk:
\begin{itemize}
  \item Online:     \url{https://imperial.service-now.com/ict/}
  \item By phone:   +44 (0)20 7594 9000
\end{itemize}




\end{document}
